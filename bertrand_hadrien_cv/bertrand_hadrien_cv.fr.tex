\documentclass[11pt,a4paper,sans]{moderncv}        % possible options include font size ('10pt', '11pt' and '12pt'), paper size ('a4paper', 'letterpaper', 'a5paper', 'legalpaper', 'executivepaper' and 'landscape') and font family ('sans' and 'roman')

% moderncv themes
\moderncvstyle{banking}                             % style options are 'casual' (default), 'classic', 'oldstyle' and 'banking'
\moderncvcolor{blue}                               % color options 'blue' (default), 'orange', 'green', 'red', 'purple', 'grey' and 'black'
%\renewcommand{\familydefault}{\sfdefault}         % to set the default font; use '\sfdefault' for the default sans serif font, '\rmdefault' for the default roman one, or any tex font name
\nopagenumbers{}                                  % uncomment to suppress automatic page numbering for CVs longer than one page

% character encoding
\usepackage[utf8]{inputenc}
\usepackage[T1]{fontenc}
\usepackage[francais]{babel}

\usepackage{outlines}

% adjust the page margins
\usepackage[top=1.1cm, bottom=1.1cm, left=2cm, right=2cm]{geometry}
%\setlength{\hintscolumnwidth}{3cm}                % if you want to change the width of the column with the dates
%\setlength{\makecvtitlenamewidth}{10cm}           % for the 'classic' style, if you want to force the width allocated to your name and avoid line breaks. be careful though, the length is normally calculated to avoid any overlap with your personal info; use this at your own typographical risks...

% personal data
\name{Hadrien}{Bertrand}              
\address{206-7221 rue Clark, H2R 0A9 Montréal}{Canada}
\email{bertrand.hadrien@gmail.com}
\phone{438 630-9626}
% \homepage{www.johndoe.com}                        
\social[linkedin]{hadrienbertrand}
\social[github]{hbertrand}                            % optional, remove / comment the line if not wanted
% \extrainfo{additional information}                 % optional, remove / comment the line if not wanted

\begin{document}
\makecvtitle

\section{Expériences Professionnelles}

\cventry{Octobre 2020 -- Présent}{Scientifique en recherche appliquée, sénior}{Mila}{Montréal}{}{}
\vspace{-16pt}
\cventry{Février 2019 -- Octobre 2020}{Scientifique en recherche appliquée}{}{}{}{}
\begin{outline}
 \1 Projets de recherche en apprentissage profond en collaboration avec l'industrie: rencontres avec les clients, montage de projets, conception de la solution, écriture du code, gestion des expériences, livraison et documentation du prototype. Exemples:
   \2 Prévision de l’irradiance solaire totale au Québec et dans le nord-est des États-Unis de zéro à six heures à l’avance en utilisant les images d'un satellite environnemental géostationnaire.
   \2 Détection de faux documents d'identité, modifiés par ordinateur. Les régions altérées étaient de taille et contenu variables, ainsi que les documents.
 \1 Accompagnement de startups dans des projets d'apprentissage automatique dans le cadre du programme PARI du CNRC.
 \1 Recrutement et gestion de stagiaires.
\end{outline}

\cventry{Novembre 2015 -- Décembre 2018}{Doctorat}{Philips Research Medisys}{Paris}{}{Thèse autour des méthodes d'apprentissage profond et de transfert d'apprentissage appliquées à des problématiques d'imagerie médicale, en particulier la classification de champ de vue en IRM et la segmentation du rein en ultrasons 3D.}

\cventry{Février 2015 -- Juillet 2015}{Stage de fin d'études}{Institut de Neurosciences de la Timone}{Marseille}{}{Validation d'une méthode d'apprentissage automatique récemment développé à l'institut. Cette méthode convertit des données d'IRM fonctionnelles en graphes, qui sont ensuites classifiés par une SVM utilisant un kernel spécialisé.}

\cventry{Janvier 2014 -- Juin 2014}{Stage, analyse automatique de code de terminaux de paiement}{Ingenico, départment R\&D}{Valence}{}{}
% Etude des techniques modernes d'analyse de code afin d'évaluer la faisabilité de la vérification de règles spécifiques aux applications sur les terminaux de paiement d'Ingenico, puis développement d'un prototype.

\cventry{Juin 2012 -- Aout 2012}{Stage, développement Android et iPhone}{Kizeo}{Avignon}{}{}
% Développement d'une application Android de présentation d'un catalogue de produits, avec gestion via un site Web. Travail sur diffèrentes applications iPhone et Android.

\cventry{Novembre 2011 -- Mars 2012}{Freelance, développement Android}{Veolia Transport Valence}{Valence}{}{}
% Développement d'une application pour permettre aux usagers sous Android d'avoir accès au plan du réseau et aux horaires des bus de Valence.


\section{Formations}
\cventry{Novembre 2015 -- Janvier 2019}{Doctorat, apprentissage profond et imagerie médicale}{Université Paris-Saclay - Télécom ParisTech - LTCI}{Paris}{}{}
\cventry{Aout 2014 -- Janvier 2015}{Semestre d'échange, Intelligence Artificielle et Apprentissage Automatique}{KTH Royal Institute of Technology}{Stockholm}{}{}
\cventry{Septembre 2010 -- Juin 2015}{Diplôme d'ingénieur, Informatique et Réseaux}{Grenoble INP ESISAR}{Valence}{}{}
\cventry{Septembre 2007 -- Juin 2010}{BAC Scientifique, Spécialité Mathématiques}{Lycée Saint-Joseph}{Avignon}{}{}


\section{Langues}
\cvitemwithcomment{Français}{Langue natale}{}
\cvitemwithcomment{Anglais}{Lu, écrit, parlé}{Score TOEIC: 990/990 - Niveau C1}


\section{Compétences en informatique}
\cvitemwithcomment{Programmation}{Python, C, PyTorch}{}
\cvitemwithcomment{IA et Apprentissage Automatique}{apprentissage profond, vision par ordinateur,}{}
\cvitemwithcomment{}{optimisation bayésienne, classification, segmentation sémantique, LLMs}{}
\cvitemwithcomment{Divers}{Unix, SQL, LaTeX, Git}{}

\section{Publications}

\begin{itemize}
    \item \textbf{(2022)} J. P. Cohen, [...], \textbf{H. Bertrand} - TorchXRayVision: A library of chest X-ray datasets and models. \textit{Published at MIDL 2022.}
    \item \textbf{(2020)} M. Hashir, \textbf{H. Bertrand}, J. P. Cohen - Quantifying the Value of Lateral Views in Deep Learning for Chest X-rays. \textit{Published at MIDL 2020.}
    \item \textbf{(2020)} J. P. Cohen, M. Hashir, R. Brooks, \textbf{H. Bertrand} - On the limits of cross-domain generalization in automated X-ray prediction. \textit{Published at MIDL 2020.}
    \item \textbf{(2019)} \textbf{H. Bertrand}, M. Hashir, J. P. Cohen - Do Lateral Views Help Automated Chest X-ray Predictions? \textit{Published at MIDL 2019.}
    \item \textbf{(2019)} \textbf{H. Bertrand} - Hyper-parameter optimization in deep learning and transfer learning: applications to medical imaging. \textit{PhD thesis.}
    \item \textbf{(2017)} \textbf{H. Bertrand}, R. Ardon, M. Perrot, I. Bloch - Hyperparameter Optimization of Deep Neural Networks: Combining Hyperband with Bayesian Model Selection. \textit{Published at CAp 2017.}
    \item \textbf{(2017)} \textbf{H. Bertrand}, M. Perrot, R. Ardon, I. Bloch - Classification of MRI data using deep learning and Gaussian process-based model selection. \textit{Published at ISBI 2017.}
\end{itemize}

\end{document}
